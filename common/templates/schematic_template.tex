\documentclass{standalone}
\usepackage{pst-optexp}

\begin{document}

% Template for optical schematic
\begin{pspicture}(-2,-2)(12,6)
    % Node definitions - Define points for components
    \pnodes(0,2){SourcePoint}(5,2){MidPoint}(10,2){EndPoint}

    \begin{optexp}
        % Components - Define your optical components here
        % Examples:
        % - Light source
        \optbox[position=start, innerlabel, optboxwidth=1.2](SourcePoint)(MidPoint){Light Source\\Type}
        
        % - Optical component (lens, mirror, beam splitter, etc.)
        % \lens(MidPoint)(EndPoint){Lens}
        % \mirror[mirrortype=extended](MidPoint)(EndPoint){Mirror}
        % \beamsplitter(MidPoint)(EndPoint){BS}
        
        % - Detector or endpoint
        \optbox[position=end, optboxwidth=1.2](EndPoint)(EndPoint){Detector}
        
        % Beam - Define the light path
        \addtopsstyle{Beam}{linestyle=none, fillstyle=solid, fillcolor=red}
        \drawwidebeam[beamwidth=0.1]{1-3}
        
        % Optional: Add labels, measurements, or annotations
        % \naput[nrot=:U]{Label}
    \end{optexp}
    
    % Optional: Add title or other annotations outside the optexp environment
    % \rput[c](5,-1.5){\Large\textbf{Title of Schematic}}
\end{pspicture}

\end{document} 